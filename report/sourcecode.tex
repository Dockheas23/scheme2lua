\label{sec:sourcecode}
This chapter contains the project source files. They are split into two parts:
the actual translator itself, labelled as the ``Core'' of the program; and the
part that was used for unit testing the two main components, which are contained
in the ``Unit Tests'' section below.

\section{Core}
\subsection{scheme2lua (Shell Script)}

\scriptsize

\begin{verbatim}
#!/bin/bash

cat dataTypes.lua
cat procedures.lua
for inputFile in $@; do
    LUA_PATH='?.lua;'${LUA_PATH} lua scheme2lua.lua < $inputFile
done
\end{verbatim}

\subsection{scheme2lua.lua}
\begin{verbatim}
require("parser")
require("procedures")
require("scanner")

-- Output the preamble
-- io.input("procedures.lua")
-- io.write(io.read("*a"))
io.input(io.stdin)

-- The main loop for the translator
scanner.init()
local expression = parser.parse(scanner.nextToken())
while expression ~= nil do
    print(tostring(expression))
    expression = parser.parse(scanner.nextToken())
end
\end{verbatim}

\subsection{scanner.lua}
\begin{verbatim}
module(..., package.seeall)

local stripLineComment, stripBlockComment, stripDatumComment
local charCache
local tokenCache

--
-- Return true if 'x' is a whitespace character
--
local function isWhiteSpace(x)
    return string.match(x, "^[%s]$")
end

--
-- Return true if 'x' is a delimiter
-- (For symbols, numbers, characters, booleans and the dotted pair marker)
--
local function isDelimiter(x)
    return isWhiteSpace(x) or x == ""
    or string.match(x, "^[%[%]%(%)\"';]$")
end

--
-- Return true if 'x' is a valid first character for a symbol
-- TODO Escaped Hex and Unicode characters currently not supported
--
local function isInitialSymbolChar(x)
    return string.match(x, "^[%a!&/:<=>~_%^%$%*%?%%]$")
end

--
-- Return true if 'x' is a valid subsequent character for a symbol
-- TODO Escaped Hex and Unicode characters currently not supported
--
local function isSubsequentSymbolChar(x)
    return isInitialSymbolChar(x) or string.match(x, "^[%d%.%+%-@]$")
end

--
-- Set up the scanner
--
function init()
    charCache = nil
    tokenCache = nil
end

--
-- Remove and return the next input character
--
local function nextChar()
    local result
    if charCache then
        result = charCache
        charCache = nil
    elseif not io.read(0) then
        result = ""
    else
        result = io.read(1)
    end
    return result
end

--
-- Return the next input character without removing it
--
local function peekChar()
    if not charCache then
        charCache = nextChar()
    end
    return charCache
end

--
-- Remove and return the next token from the scanner
--
function nextToken()
    local token
    local value = nil
    local c

    if tokenCache then
        token, value = tokenCache[1], tokenCache[2]
        tokenCache = nil
        return token, value
    end

    repeat
        c = nextChar()
        if c == ";" then
            c = stripLineComment()
        elseif c == "#" and peekChar() == "|" then
            c = stripBlockComment()
        elseif c == "#" and peekChar() == ";" then
            c = stripDatumComment()
        end
    until not isWhiteSpace(c) or c == ""

    if c == "" then
        token = "EOF"

    -- Check for peculiar symbols '+' and '-'
    elseif (c == "+" or c == "-") and isDelimiter(peekChar()) then
        token = c

    -- Check for peculiar symbol '...'
    elseif c == "." and peekChar() == "."
        and nextChar() == "." and peekChar == "." then
        nextChar()
        token = "..."

    elseif c == "-" and peekChar() == ">"   -- Another peculiar symbol
        or isInitialSymbolChar(c) then      -- Regular symbol
        local symbolValue = c
        while isSubsequentSymbolChar(peekChar()) do
            c = nextChar()
            symbolValue = symbolValue .. c
        end
        token, value = "Symbol", symbolValue

    elseif c == "#" and string.match(peekChar(), "[TtFf]") then
        token = "Boolean"
        if string.match(nextChar(), "[Tt]") then
            value = true
        else
            value = false
        end

    elseif c == "#" and peekChar() == "\\" then
        --
        -- TODO Named and hex characters currently not supported
        --
        token = "Character"
        nextChar()
        value = nextChar()
        
    elseif c == "\"" then
        --
        -- TODO Escape characters and multiline strings currently not
        -- supported
        --
        local stringValue = ""
        c = nextChar()
        while c ~= "\"" and c ~= "" do
            stringValue = stringValue .. c
            c = nextChar()
        end
        token, value = "String", stringValue

    elseif c == "#" and string.match(peekChar(), "[bodxie]")
        --
        -- TODO Numbers not properly supported yet
        --
        or string.match(c, "[+%-%d]") then
        local numberValue = string.byte(c) - string.byte("0")
        while string.match(peekChar(), "%d") do
            c = nextChar()
            local digit = string.byte(c) - string.byte("0")
            numberValue = 10 * numberValue + digit
        end
        token, value = "Number", numberValue
        
    elseif c == "#" and peekChar() == "(" then
        nextChar()
        token = "#("

    elseif c == "#" and string.match(peekChar(), "[vV]")
        and nextChar() and string.match(peekChar(), "[uU]") then
        nextChar()
        nextChar()
        token = "#vu8("

    else token = c
        --
        -- TODO Checking for invalid tokens currently not supported
        --
    end

    return token, value
end

--
-- Return the next token from the scanner without removing it
--
function peekToken()
    if tokenCache then
        return tokenCache[1], tokenCache[2]
    else
        local token, value = nextToken()
        tokenCache = {token, value}
        return token, value
    end
end

--
-- Remove a line comment from the scanner and return the next character
--
stripLineComment = function ()
    local c
    repeat
        c = nextChar()
    until c == "\n" or c == ""
    return c
end

--
-- Remove a block comment from the scanner and return the next character
--
stripBlockComment = function ()
    local c
    local nestLevel = 1
    nextChar()
    repeat
        c = nextChar()
        if c == "#" and nextChar() == "|" then
            c = nextChar()
            nestLevel = nestLevel + 1
        elseif c == "|" and nextChar() == "#" then
            c = nextChar()
            nestLevel = nestLevel - 1
        end
    until nestLevel == 0 or c == ""
    return c
end

--
-- Remove a block comment from the scanner and return the next character
-- TODO Datum comments not currently supported
--
stripDatumComment = function ()
    return c
end
\end{verbatim}

\subsection{parser.lua}
\begin{verbatim}
module(..., package.seeall)

require("dataTypes")
require("forms")
require("scanner")

local applyFunction, getArguments, readCompound

--
-- Return an iterator over the arguments in a Scheme list syntactic form
--
getArguments = function ()
    return function ()
        local token, value = scanner.peekToken()
        if token == ")" or token == "]" then
            return nil
        end
        return scanner.nextToken()
    end
end

--
-- Return the set of Scheme function arguments from the scanner as a
-- string
--
gatherArguments = function (funcName)
    if forms.specialArgs[funcName] then
        return forms.specialArgs[funcName]()
    else
        local result = scmArglist:new()
        for arg, argValue in getArguments() do
            result:add(parse(arg, argValue))
        end
        if forms.wrappedArgs[funcName] then
            return result:wrappedArgs()
        end
        return result:selfAsString()
    end
end

--
-- Apply the Scheme function 'func'
--
applyFunction = function (func)
    local f = tostring(func)
    if forms.specialSyntax[f] then
        return forms.specialSyntax[f]()
    end
    return (forms.procedures[f] or f)
    .. "(" .. gatherArguments(f) .. ")"
end

--
-- Read and return a single scheme datum from the scanner
--
readDatum = function ()
    local token, value = scanner.peekToken()
    if token == "(" or token == "[" then
        scanner.nextToken()   -- Remove initial bracket
        local list = readList()
        scanner.nextToken()   -- Remove ending bracket
        return list
    elseif token == "#(" or token == "#vu8(" then
        return readVector()
    end
    scanner.nextToken()
    return parse(token, value)
end

--
-- Read and return a scheme list from the scanner (not including brackets)
--
readList = function ()
    local token = scanner.peekToken()
    if token == ")" or token == "]" then
        return scmList:new()
    elseif token == "." then
        scanner.nextToken()
        return readDatum()
    else
        return scmList:new{car = readDatum(); cdr = readList()}
    end
end

--
-- Read and return a scheme vector or bytevector from the scanner (including
-- brackets)
--
readVector = function ()
    local token = scanner.nextToken()
    local objectType = scmVector
    local value = {}
    if token == "#vu8(" then
        objectType = scmBytevector
    end
    token = scanner.peekToken()
    while token ~= ")" do
        table.insert(scmValue, readDatum())
        token = scanner.peekToken()
    end
    scanner.nextToken()
    return objectType:new(value)
end

--
-- Evaluate the Scheme token 'token', with value 'value', and return an scmData
-- encapsulating the appropriate Lua translation
--
function parse(token, value)
    if token == "EOF" then
        return nil
    end

    if token == "Boolean" then
        return scmBoolean:new(value)

    elseif token == "Character" then
        return scmCharacter:new(value)

    elseif token == "String" then
        return scmString:new(value)

    elseif token == "Number" then
        return scmNumber:new(value)

    elseif token == "Symbol" then
        return scmSymbol:new(value)

    elseif token == "(" or token == "[" then
        local procedureValue = applyFunction(parse(scanner.nextToken()))
        scanner.nextToken()
        return scmProcedure:new(procedureValue)

    elseif token == "'" then
        return readDatum()

    else
        return scmData:new(token)
    end
end
\end{verbatim}

\subsection{forms.lua}
\begin{verbatim}
module(..., package.seeall)

require("parser")
require("scanner")

--
-- A utility function that returns the string representation of a sequence of
-- scheme expressions (or "body"s) from the scanner, returning the last one.
-- This is a common syntax for lambda, case-lambda, cond, let and others
--
local function bodyList()
    local result = ""
    local token = scanner.peekToken()
    while token ~= ")" do
        local expr = parser.parse(scanner.nextToken())
        if scanner.peekToken() == ")" then
            result = result .. "return "
        end
        result = result .. expr:selfAsString() .. "\n"
        token = scanner.peekToken()
    end
    return result
end;

--
-- A table of supported special syntaxes. Each is a function that returns the
-- final output string for that syntax.
--
specialSyntax = {
    ["cond"] = function ()
        local result = "(function ()\n"
            local token = scanner.peekToken()
            local first = true
            while token ~= ")" do
                scanner.nextToken()  -- Remove initial bracket of clause
                local test = parser.parse(scanner.nextToken())
                if first then
                    result = result .. "if "
                    .. test:selfAsString() .. ".value then\n"
                    first = false
                elseif tostring(test) == "else" then
                    result = result .. "else\n"
                else
                    result = result .. "elseif "
                    .. test:selfAsString() .. ".value then\n"
                end
                local expr = scanner.peekToken()
                if expr == "=>" then
                    scanner.nextToken()
                    local func = parser.parse(scanner.nextToken())
                    result = result .. "return " .. func:selfAsString() ..
                    "(" .. test:selfAsString() .. ")\n"
                elseif expr == ")" then
                    scanner.nextToken()
                    result = result .. "return " .. test:selfAsString() .. "\n"
                else
                    result = result .. bodyList()
                    scanner.nextToken()
                end
                token = scanner.peekToken()
            end
            return result .. "end end)()\n"
        end;

    ["define"] = function ()
        -- TODO implement other forms of define
        local assignee = parser.parse(scanner.nextToken())
        local object = parser.parse(scanner.nextToken())
        return assignee:selfAsString() .. " = " .. object:selfAsString()
    end;

    ["if"] = function ()
        local result = "(function ()\nif "
            .. parser.parse(scanner.nextToken()):selfAsString()
            .. ".value then\nreturn"
            .. parser.parse(scanner.nextToken()):selfAsString() .. "\n"
        if scanner.peekToken() ~= ")" then
            result = result .. "else\nreturn"
            .. parser.parse(scanner.nextToken()):selfAsString() .. "\n"
        end
        return result .. "end end)()\n"
    end;

    ["lambda"] = function ()
        local paramList = {}
        local vararg = false
        local params = parser.readDatum()
        if params.scmType == "List" then
            while params.value ~= nil do
                table.insert(paramList, scmSymbol:new(params.value.car))
                if params.value.cdr.scmType ~= "List" then
                    vararg = params.value.cdr
                    break
                end
                params = params.value.cdr
            end
        else
            vararg = params
        end
        local result = "(function ("
            .. scmArglist.fromTable(paramList):selfAsString()
            .. (vararg and "..." or "") .. ")\n"
            if vararg ~= false then
                result = result .. "local " .. tostring(vararg)
                .. " = scmList.fromTable{...}\n"
            end
            result = result .. bodyList() .. "end)"
        return result
    end;

    ["let"] = function ()
        local funcName = nil
        if scanner.peekToken() == "Symbol" then
            funcName = tostring(parser.parse(scanner.nextToken()))
        end
        scanner.nextToken() -- Opening '(' of var list
        local varNames = {}
        local varValues = {}
        local token = scanner.peekToken()
        while token ~= ")" do
            scanner.nextToken() -- Opening '(' of var definition
            table.insert(varNames, parser.parse(scanner.nextToken()))
            table.insert(varValues, parser.parse(scanner.nextToken()))
            scanner.nextToken() -- Closing ')' of var definition
            token = scanner.peekToken()
        end
        scanner.nextToken() -- Closing ')' of var list
        local funcDef = "(function "
            .. "(" .. scmArglist.fromTable(varNames):selfAsString() .. ")\n"
            .. bodyList() .. "end)"
        local args = scmArglist.fromTable(varValues):selfAsString()
        if funcName then
            return "(function ()\n" .. funcName .. " = " .. funcDef
                .. "\nreturn " .. funcName .. "(" .. args .. ") end)()\n"
        else
            return funcDef .. "(" .. args .. ")\n"
        end
    end;
}

--
-- A table for scheme functions that need all of their arguments wrapped in
-- functions. This prevents the arguments from being fully evaluated and is
-- useful for short-circuit operations
--
wrappedArgs = {
    ["and"] = true;
    ["or"] = true;
}

--
-- A table for scheme functions that need their arguments collected in a
-- non-standard way. Each is a function that returns the argument list as a
-- string.
--
specialArgs = {
    ["quote"] = function ()
        return parser.readDatum():selfAsString()
    end;
}

--
-- A table of the supported Scheme procedures
--
procedures = {
    ["cond"] = "s2l_cond";
    ["quote"] = "s2l_quote";

    ["cons"] = "s2l_cons";
    ["car"] = "s2l_car";
    ["cdr"] = "s2l_cdr";
    ["list"] = "s2l_list";
    ["length"] = "s2l_length";
    ["filter"] = "s2l_filter";

    -- Useful output routines
    ["display"] = "s2l_display";
    ["newline"] = "s2l_newline";

    -- Arithmetic operators
    ["+"] = "s2l_arithmeticPlus";
    ["-"] = "s2l_arithmeticMinus";
    ["*"] = "s2l_arithmeticMultiply";
    ["/"] = "s2l_arithmeticDivide";

    -- Relational operators
    ["<"] = "s2l_lessThan";
    ["<="] = "s2l_lessThanOrEqual";
    ["="] = "s2l_equals";
    [">"] = "s2l_greaterThan";
    [">="] = "s2l_greaterThanOrEqual";
    ["eqv?"] = "s2l_equals";

    -- Boolean operators
    ["and"] = "s2l_and";
    ["or"] = "s2l_or";
    ["not"] = "s2l_not";

    -- Predicates
    ["pair?"] = "s2l_pair";
    ["boolean?"] = "s2l_boolean";
    ["null?"] = "s2l_null";
    ["number?"] = "s2l_number";
    ["integer?"] = "s2l_integer";

    -- Exit
    ["exit"] = "s2l_exit";
    ["quit"] = "s2l_exit";
}
\end{verbatim}

\subsection{dataTypes.lua}
\begin{verbatim}
--
-- SCHEME TO LUA TRANSLATOR
-- Preamble File 1
-- This file contains lua container tables for scheme data
--

--
-- Container for basic scheme data. Actual types inherit from this.
-- Noteworthy methods:
-- selfAsString - an (overrideable) object function that returns a string of
-- lua code that will recreate that object
-- valueAsString - an (overrideable) object function that returns the output
-- representation of that scheme object
--
scmData = {
    scmType = "Data";
    create = function (self, o)
        local result = o or {}
        setmetatable(result, self)
        self.__index = self
        self.__tostring = self.valueAsString or scmData.valueAsString
        return result
    end;
    new = function (self, item)
        return self:create{value = item}
    end;
    selfAsString = function (self)
        local funcName = "scm" .. self.scmType .. ":new"
        return funcName .. "("..tostring(self.value)..")"
    end;
    valueAsString = function (self)
        return tostring(self.value)
    end;
}

--
-- This wraps scheme function arguments with an iterative interface
--
scmArglist = scmData:create{
    scmType = "Arglist";
    currentIndex = 1;
    numElements = 0;
    add = function (self, item)
        if self.value == nil then
            self.value = {}
        end
        self.numElements = self.numElements + 1
        self.value[self.numElements] = item
    end;
    hasNext = function (self)
        return self.currentIndex > self.numElements
    end;
    nextArg = function (self)
        local result = self.value[self.currentIndex]
        self.currentIndex = self.currentIndex + 1
        return result
    end;
    itemsAsString = function (self)
        local result = ""
        for _, item in ipairs(self.value) do
            result = result .. item:selfAsString() .. ", "
        end
        if result ~= "" then
            result = string.sub(result, 1, -3)
        end
        return result
    end;
    selfAsString = function (self)
        if self.numElements == 0 then
            return ""
        end
        return self:itemsAsString()
    end;
    wrappedArgs = function (self)
        if self.numElements == 0 then
            return ""
        end
        local result = ""
        for _, item in ipairs(self.value) do
            result = result .. "(function () return " .. item:selfAsString()
                .. " end), "
        end
        if result ~= "" then
            result = string.sub(result, 1, -3)
        end
        return result
    end;
    fromTable = function (table)
        local result = scmArglist:new()
        for _, value in ipairs(table) do
            result:add(value)
        end
        return result
    end;
}

--
-- Data type to hold a scheme procedure
--
scmProcedure = scmData:create{
    scmType = "Procedure";
    selfAsString = function (self)
        return tostring(self.value);
    end;
}

--
-- Data type to hold a scheme symbol
--
scmSymbol = scmData:create{
    scmType = "Symbol";
    selfAsString = function (self)
        return self:valueAsString();
    end;
}

--
-- Data type to hold a scheme boolean
--
scmBoolean = scmData:create{
    scmType = "Boolean";
    valueAsString = function (self)
        return "#" .. (self.value and "t" or "f")
    end;
}

--
-- Data type to hold a scheme character
--
scmCharacter = scmData:create{
    scmType = "Character";
    selfAsString = function (self)
        return "scmCharacter:new(\"" .. self.value .. "\")"
    end;
}

--
-- Data type to hold a scheme string
--
scmString = scmData:create{
    scmType = "String";
    selfAsString = function (self)
        return "scmString:new(\"" .. self.value .. "\")"
    end;
}

--
-- Data type to hold a scheme number
--
scmNumber = scmData:create{
    scmType = "Number";
}

--
-- Data type to hold a scheme list
--
scmList = scmData:create{
    scmType = "List";
    fromTable = function (table)
        return scmList.fromTableIndex(table, 1)
    end;
    fromTableIndex = function (table, currentIndex)
        if table[currentIndex] == nil then
            return scmList:new()
        elseif tostring(table[currentIndex]) == "." then
            return table[currentIndex + 1]
        else
            return scmList:new{
                car = table[currentIndex];
                cdr = scmList.fromTableIndex(table, currentIndex + 1)
            }
        end
    end;
    fromArglist = function (args)
        return scmList.fromTableIndex(args.value, args.currentIndex)
    end;
    itemsAsString = function (self)
        if self.value == nil then
            return ""
        end
        local car = self.value.car
        local cdr = self.value.cdr
        if cdr.scmType ~= "List" then
            return tostring(car) .. " . " .. tostring(cdr)
        elseif cdr.value == nil then
            return tostring(car)
        else
            return tostring(car) .. " " .. cdr:itemsAsString()
        end
    end;
    selfAsString = function (self)
        if self.value == nil then
            return "scmList:new()"
        end
        return "scmList:new{\ncar = " .. self.value.car:selfAsString()
        .. ";\ncdr = " .. self.value.cdr:selfAsString() .. "}"
    end;
    valueAsString = function (self)
        return "(" .. self:itemsAsString() .. ")"
    end;
}

--
-- Data type to hold a scheme vector
--
scmVector = scmData:create{
    scmType = "Vector";
    itemsAsString = function (self)
        local result = ""
        for _, v in ipairs(self.value) do
            result = result .. tostring(v) .. " "
        end
        if result ~= "" then
            result = string.sub(result, 1, -2)
        end
        return result
    end;
    valueAsString = function (self)
        return "#(" .. self:itemsAsString() .. ")"
    end;
}

--
-- Data type to hold a scheme bytevector
--
scmBytevector = scmVector:create{
    scmType = "Bytevector";
    valueAsString = function (self)
        return "#vu8(" .. self:itemsAsString() .. ")"
    end;
}
\end{verbatim}

\subsection{procedures.lua}
\begin{verbatim}
--
-- SCHEME TO LUA TRANSLATOR
-- Preamble File 2
-- This file contains the lua implementation of the scheme functions
--

--
-- BEGIN SCHEME FUNCTIONS
--

function s2l_quote(obj)
    return obj
end

function s2l_cons(obj1, obj2)
    return scmList:new{car = obj1; cdr = obj2}
end

function s2l_car(pair)
    return pair.value.car
end

function s2l_cdr(pair)
    return pair.value.cdr
end

function s2l_list(...)
    return scmList.fromTable{...}
end

function s2l_length(list)
    local result = 0
    while not s2l_null(list).value do
        result = result + 1
        list = s2l_cdr(list)
    end
    return scmNumber:new(result)
end

function s2l_filter(proc, list)
    if s2l_null(list).value then
        return scmList:new()
    elseif proc(s2l_car(list)).value then
        return s2l_cons(s2l_car(list), s2l_filter(proc, s2l_cdr(list)))
    else
        return s2l_filter(proc, s2l_cdr(list))
    end
end

function s2l_display(obj)
    io.write(tostring(obj))
end

function s2l_newline()
    io.write("\n")
end

function s2l_arithmeticPlus(...)
    local result = 0
    for _, item in ipairs{...} do
        result = result + item.value
    end
    return scmNumber:new(result)
end

function s2l_arithmeticMinus(a, b)
    return scmNumber:new(a.value - b.value)
end

function s2l_arithmeticMultiply(...)
    local result = 1
    for _, item in ipairs{...} do
        result = result * item.value
    end
    return scmNumber:new(result)
end

function s2l_arithmeticDivide(a, b)
    return scmNumber:new(a.value / b.value)
end

function s2l_lessThan(n1, n2)
    return scmBoolean:new(n1.value < n2.value)
end

function s2l_lessThanOrEqual(n1, n2)
    return scmBoolean:new(n1.value <= n2.value)
end

function s2l_equals(n1, n2)
    return scmBoolean:new(n1.value == n2.value)
end

function s2l_greaterThan(n1, n2)
    return scmBoolean:new(n1.value > n2.value)
end

function s2l_greaterThanOrEqual(n1, n2)
    return scmBoolean:new(n1.value >= n2.value)
end

-- Short circuit
function s2l_and(...)
    local result = scmBoolean:new(true)
    for _, f in ipairs{...} do
        local item = f()
        if item.value == false then
            return scmBoolean:new(false)
        else
            result = item
        end
    end
    return result
end

-- Short circuit
function s2l_or(...)
    for _, f in ipairs{...} do
        local item = f()
        if item.value then
            return item
        end
    end
    return scmBoolean:new(false)
end

function s2l_not(obj)
    if obj.value == false then
        return scmBoolean:new(true)
    end
    return scmBoolean:new(false)
end

function s2l_pair(obj)
    return scmBoolean:new(obj.scmType == "List" and obj.value ~= nil)
end

function s2l_boolean(obj)
    return scmBoolean:new(obj.scmType == "Boolean")
end

function s2l_null(obj)
    return scmBoolean:new(obj.scmType == "List" and obj.value == nil)
end

function s2l_number(obj)
    return scmBoolean:new(obj.scmType == "Number")
end

function s2l_integer(obj)
    return scmBoolean:new(obj.scmType == "Number" and obj.value % 1 == 0)
end

function s2l_exit()
    return nil
end

--
-- END SCHEME FUNCTIONS
--
\end{verbatim}


\section{Unit Tests}
\subsection{run-tests (Shell Script)}
\begin{verbatim}
#!/bin/bash

for testsuite in tests/*.lua; do
    LUA_PATH='?.lua;tests/?.lua;'${LUA_PATH} lua $testsuite
done
\end{verbatim}

\subsection{scannertest.lua}
\begin{verbatim}
require("scanner")

--
-- Create test input file
--
io.output("/tmp/s2ltest_chars.scm")
io.write("(+ 1 2)\n")
io.write("(display \"abc\")\n")
io.close()

io.output("/tmp/s2ltest_tokens.scm")
io.write("(newline) ; This should print a newline\n")
io.write("#| Commented out section\n")
io.write("(+ 1 2)\n")
io.write("(define x #| Nested #| Another nest |# block comment |# 5)\n")
io.write("(display \"abc\")\n")
io.write("End of commented out section |#\n")
io.write("; Another one-line comment|#\n")
io.write("(lambda (->x) (+ ->x 1))\n")
io.write("(* (+ 3 5) 2 3)\n")
io.close()

io.output(io.stdout)

characters = {
    "(", "+", " ", "1", " ", "2", ")", "\n",
    "(", "d", "i", "s", "p", "l", "a", "y", " ",
    "\"", "a", "b", "c", "\"", ")", "\n", 
}

tokens = {
    {"("}, {"symbol", "newline"}, {")"},
    {"("}, {"symbol", "lambda"},
	{"("}, {"symbol", "->x"}, {")"},
	{"("}, {"+"}, {"symbol", "->x"}, {"number", 1}, {")"},
    {")"},
    {"("}, {"symbol", "*"},
	{"("}, {"+"}, {"number", 3}, {"number", 5}, {")"},
	{"number", 2}, {"number", 3},
    {")"}
}

--
-- Run tests
--

-- Test char functions
io.input("/tmp/s2ltest_chars.scm")
scanner.init()
local index = 1
while scanner.peekChar() ~= "" do
    assert(scanner.peekChar() == characters[index]
    and scanner.nextChar() == characters[index])
    index = index + 1
end

-- Test token functions
io.input("/tmp/s2ltest_tokens.scm")
scanner.init()
index = 1
while scanner.peekToken() ~= "EOF" do
    local token, value = scanner.peekToken()
    assert(token == tokens[index][1] and value == tokens[index][2])
    token, value = scanner.nextToken()
    assert(token == tokens[index][1] and value == tokens[index][2])
    index = index + 1
end

print("Scannertest: All tests passed")
\end{verbatim}

\subsection{parsertest.lua}
\begin{verbatim}
require("parser")
require("scanner")

--
-- Create test input file
--
io.output("/tmp/s2ltest_parser.scm")
io.write("display \"abc\"\n")
io.write("123 #\\l\n")
io.write("#f #t")
io.write("5 (\"a\" (1 2 3) #t #\\m)")
io.close()

io.output(io.stdout)

--
-- Run tests
--
io.input("/tmp/s2ltest_parser.scm")
scanner.init()
if assert(parser.evaluate(scanner.peekToken()) == "display") 
    and assert(parser.evaluate(scanner.nextToken()) == "display") 
    and assert(parser.evaluate(scanner.nextToken()) == "\"abc\"") 
    and assert(parser.evaluate(scanner.nextToken()) == 123) 
    and assert(parser.evaluate(scanner.nextToken()) == "\"l\"") 
    and assert(parser.evaluate(scanner.nextToken()) == false) 
    and assert(parser.evaluate(scanner.nextToken()) == true) 
    and assert(parser.readDatum() == 5) 
    and assert(parser.readDatum()[2][3] == 3) 
    then
    print("Parsertest: All tests passed")
end
\end{verbatim}

\normalsize
