Testing was a constant process, with each new feature invariably requiring
thorough inspection. As the project progressed, the testing strategy graduated
from unit tests to more extensive general tests. Unit tests were performed on
the two main components of the translator. Once they were working properly,
testing consisted mainly of verifying various features through a series of
example Scheme programs.


\section{Unit Tests}
Unit tests were used to iron out the early bugs in the scanner and parser. They
were in the form of Lua scripts. As an example, here is the script used to test
the scanner by creating temporary test scheme files and verifying the characters
and tokens produced by the scanner.
\begin{framed}
\scriptsize
\begin{verbatim}
require("scanner")

--
-- Create test input file
--
io.output("/tmp/s2ltest_chars.scm")
io.write("(+ 1 2)\n")
io.write("(display \"abc\")\n")
io.close()

io.output("/tmp/s2ltest_tokens.scm")
io.write("(newline) ; This should print a newline\n")
io.write("#| Commented out section\n")
io.write("(+ 1 2)\n")
io.write("(define x #| Nested #| Another nest |# block comment |# 5)\n")
io.write("(display \"abc\")\n")
io.write("End of commented out section |#\n")
io.write("; Another one-line comment|#\n")
io.write("(lambda (->x) (+ ->x 1))\n")
io.write("(* (+ 3 5) 2 3)\n")
io.close()

io.output(io.stdout)

characters = {
    "(", "+", " ", "1", " ", "2", ")", "\n",
    "(", "d", "i", "s", "p", "l", "a", "y", " ",
    "\"", "a", "b", "c", "\"", ")", "\n", 
}

tokens = {
    {"("}, {"symbol", "newline"}, {")"},
    {"("}, {"symbol", "lambda"},
	{"("}, {"symbol", "->x"}, {")"},
	{"("}, {"+"}, {"symbol", "->x"}, {"number", 1}, {")"},
    {")"},
    {"("}, {"symbol", "*"},
	{"("}, {"+"}, {"number", 3}, {"number", 5}, {")"},
	{"number", 2}, {"number", 3},
    {")"}
}

--
-- Run tests
--

-- Test char functions
io.input("/tmp/s2ltest_chars.scm")
scanner.init()
local index = 1
while scanner.peekChar() ~= "" do
    assert(scanner.peekChar() == characters[index]
    and scanner.nextChar() == characters[index])
    index = index + 1
end

-- Test token functions
io.input("/tmp/s2ltest_tokens.scm")
scanner.init()
index = 1
while scanner.peekToken() ~= "EOF" do
    local token, value = scanner.peekToken()
    assert(token == tokens[index][1] and value == tokens[index][2])
    token, value = scanner.nextToken()
    assert(token == tokens[index][1] and value == tokens[index][2])
    index = index + 1
end

print("Scannertest: All tests passed")
\end{verbatim}
\end{framed}


\section{Scheme Test Programs}

Below are some excerpts from the programs that were used. They were made to be
increasingly complex and to test different aspects of the translator. A more
complete example, including the translation and program output can be found in
Appendix~\ref{sec:transexample}

\subsection{Test Program 1}
This example tested basic operations, such as displaying output, quoting data
and constructing proper and improper lists.
\begin{framed}
\begin{verbatim}
(display '())
(newline)
(display (quote (1 #t 3)))
(newline)
(display '(1 2 . 3))
(newline)
(display (cons 4 (cons '("Hello" 3) '(1 2 . 3))))
(newline)
\end{verbatim}
\end{framed}

\subsection{Test Program 2}
This example tested simple definitions and function application.
\begin{framed}
\begin{verbatim}
(define identity (lambda (x) x))
(display (identity 5))

(display ((lambda (x) (+ x 1)) 2))
(newline)

(define a
  (lambda (n) (+ n 5)))
(define b
  (lambda (x y) (+ x y)))

(display "(b (a 3) (b 2 2))")
(newline)
(display (b (a 3) (b 2 2)))
(newline)
\end{verbatim}
\end{framed}

\subsection{Test Program 3}
This program was used to test the different ways of handling arguments in the
lambda expression.
\begin{framed}
\begin{verbatim}
(display ((lambda (x) x) 5))
(newline)
(display ((lambda x x) 5))
(newline)
(display ((lambda (x . y) y) 1 2 3))
(newline)
\end{verbatim}
\end{framed}

\subsection{Test Program 4}
This program tested some more comprehensive function definitions, including
branching and recursion.
\begin{framed}
\begin{verbatim}
(define IsInteger
  (lambda (x)
    (integer? x)))

(define Last
  (lambda (l)
    (cond
      ((null? (cdr l)) (car l))
      (else (Last (cdr l))))))

(define Range
  (lambda (lo hi)
    (cond
      ((> lo hi) '())
      (else (cons lo (Range (+ lo 1) hi))))))

(display (Range 1 5))
(display (Last '(9 8 7 6 5 4)))
\end{verbatim}
\end{framed}


\section{Performance Tests}

Two prime number algorithms were used to test the relative performance of the
translation against a native Scheme implementation. Also included is the result
of running the translated Lua output using \texttt{luaJIT}, the Lua just-in-time
compiler. The algorithms were intentionally na\"{i}ve, with the second being
a more list-based variation. A comparison with a native Lua version of
Algorithm 1 is also included.

These tests were all conducted on the same computer, in the same environment,
using the UNIX \texttt{time} utility. The real time value was taken and rounded
to the nearest second to give the results below.

\subsection{Algorithm 1}

This algorithm generates the first $n$ prime numbers by checking every number in
sequence, starting at 2, and outputting it if it is prime. The sequence stops
when $n$ primes are output. To determine if a number, $k$, is prime, it attempts
to divide $k$ by all numbers from $2$ to $k - 1$ in sequence. If none divide
evenly, then $k$ is considered prime.

\begin{framed}
\begin{verbatim}
(define primes
  (lambda (n)
    (let nextprime ((primecount 1) (current 2))
      (cond
        ((> primecount n) (newline))
        ((isPrime current)
         (display primecount) (display ": ")
         (display current) (newline)
         (nextprime (+ primecount 1) (+ current 1)))
        (else (nextprime primecount (+ current 1)))))))

(define isPrime
  (lambda (n)
    (if (< n 2) #f
      (let prime ((k 2))
        (cond
          ((= n k) #t)
          ((integer? (/ n k)) #f)
          (else (prime (+ k 1))))))))

(primes 5000)
\end{verbatim}
\end{framed}

\subsubsection{Results}

\begin{center}
\begin{tabular}{|l|c|c|c|}
\hline
& lua & luaJIT & scm \\
\hline
First 1000 Primes & 0:16 & 0:09 & 0:02 \\ \hline
First 2000 Primes & 1:13 & 0:41 & 0:09 \\ \hline
First 5000 Primes & 8:26 & 4:49 & 1:02 \\ \hline
\end{tabular}
\end{center}
This test shows that the native Scheme interpreter, running the native Scheme
code, is substantially faster than the translation, at an approximate ratio of
$8:4:1$. What is interesting is the degree to which \texttt{luaJIT} is faster
than \texttt{lua} on the same translated code.

These results are probably to be expected, given that the design of the Scheme
interpreter is likely optimised for the Scheme style of programming, in a way
that the Lua interpreters would not be. Also, creating Scheme-like wrappers for
the data in Lua may have nullified any speed advantages that native Lua might
have had over Scheme.

\subsection{Algorithm 2}

This algorithm generates all prime numbers up to a number, $n$. It does this by
filtering the predicate $isPrime$ over the list of all numbers from 1 to $n$.
$isPrime$ uses the same method as the Algorithm 1 to determine if a number is
prime.
\begin{framed}
\begin{verbatim}
(define Range
  (lambda (lo hi)
    (cond
      ((> lo hi) '())
      (else (cons lo (Range (+ lo 1) hi))))))

(define Filter
  (lambda (p l)
    (cond
      ((null? l) '())
      ((p (car l)) (cons (car l) (Filter p (cdr l))))
      (else (Filter p (cdr l))))))

(define isPrime
  (lambda (n)
    (if (< n 2) #f
      (let prime ((k 2))
        (cond
          ((= n k) #t)
          ((integer? (/ n k)) #f)
          (else (prime (+ k 1))))))))

(display (Filter isPrime (Range 2 20000)))
(newline)
\end{verbatim}
\end{framed}

\subsubsection{Results}

\begin{center}
\begin{tabular}{|l|c|c|c|}
\hline
& lua & luaJIT & scm \\
\hline
Primes Up To 10000 & 0:37 & 0:16 & 0:03 \\ \hline
Primes Up To 20000 & FAIL & 1:13 & FAIL \\ \hline
\end{tabular}
\end{center}
The figures for the primes up to 10000 show a similar profile as before, but
with the ratio now at around $12:5:1$. The failure of both \texttt{lua} and
\texttt{scm} on the primes up to 20000 was an interesting and unexpected result.
In the case of the \texttt{lua}, it was caused by a stack overflow---a
consequence of trying to build the list of numbers from 1 to 20000. For
\texttt{scm}, it was an unspecified segmentation fault, which could be due to
similar over-consumption of memory, or a bug in the Scheme interpreter.
\texttt{luaJIT} was able to complete the task without problems.

\subsection{Native Lua}

As a final comparison, the translated results were compared against a version of
Algorithm 1 written in native Lua and run through the regular Lua interpreter.
Here is the Lua program used:

\begin{framed}
\begin{verbatim}
function primes(n)
    local primecount = 1
    local current = 2
    repeat
        if primecount > n then
            print("\n")
            break
        elseif isPrime(current) then
            print(primecount .. ": " .. current)
            primecount = primecount + 1
            current = current + 1
        else
            current = current + 1
        end
    until false
end

function isPrime(n)
    if n < 2 then
        return false
    else
        local k = 2
        repeat
            if n == k then
                return true
            elseif n % k == 0 then
                return false
            else
                k = k + 1
            end
        until false
    end
end

primes(5000)
\end{verbatim}
\end{framed}

\subsubsection{Results}

Adding the recorded times to the results from Algorithm 1 gives:
\begin{center}
\begin{tabular}{|l|c|c|c|c|}
\hline
& lua & luaJIT & scm & lua (native) \\
\hline
First 1000 Primes & 0:16 & 0:09 & 0:02 & $<$0:01 \\ \hline
First 2000 Primes & 1:13 & 0:41 & 0:09 & 0:01 \\ \hline
First 5000 Primes & 8:26 & 4:49 & 1:02 & 0:07 \\ \hline
\end{tabular}
\end{center}
This displays the staggering speed of Lua for code written specifically for it.
It may not be an entirely fair comparison---the particulars of the algorithm
had to differ slightly due to the difference between Scheme's and Lua's looping
constructs---but the difference is surprising nonetheless.
