\section{Methodology}


\section{Project Management}

Given the experimental nature of the project, and the fact that it was a single
person project, there was no fixed schedule and formal project management
methodologies were not rigorously applied. It was also clear from the beginning
that it would not be possible to have a thoroughly complete translator in the
available time. As a result, the project followed the course of a prioritised
list of objectives, which were followed in a serial fashion for the most part.
Certain minor features were worked on in parallel, and the source control
management system facilitated that, but this was largely the exception for two
reasons: firstly each feature needed to be completed in full to avoid breaking
the entire program; and secondly, there was significant linear dependency
between features.


\section{Tools}

\subsection{Text Editor}

\subsection{Source Control Management}

Bazaar\footnote{http://bazaar-vcs.org/}

Git\footnote{http://git-scm.com/}

\subsection{Typesetter}

This report was written and typeset using \LaTeX.


\section{Supervisor Meetings}

The supervisor meetings were held in Dr.\ Manning's office on a weekly basis,
usually on Thursday afternoons. The meetings generally had the following format:

\begin{enumerate}
\item An investigation of some specific problems encountered
\item A general discussion of the current state of the project, and the
features of Scheme and Lua
\item A look at the project direction and goals in high-level terms
\end{enumerate}

The frequency of the meetings helped the momentum of the project, and it meant
that potential pitfalls could be confronted and dealt with on a regular basis.
Revisiting and revising objectives every week made it flexibile and dynamic,
which facilitated exploration of specific areas when necessary.

