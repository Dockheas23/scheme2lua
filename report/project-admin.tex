\section{Methodology}


\section{Project Management}

Given the experimental nature of the project, and the fact that it was a single
person project, there was no fixed schedule and formal project management
methodologies were not rigorously applied. It was also clear from the beginning
that it would not be possible to have a thoroughly complete translator in the
available time. As a result, the project followed the course of a prioritised
list of objectives, which were followed in a serial fashion for the most part.
Certain minor features were worked on in parallel, and the source code
management system facilitated that, but this was largely the exception for two
reasons: firstly each feature needed to be completed in full to avoid breaking
the entire program; and secondly, there was significant linear dependency
between features.


\section{Tools}

\subsection{Text Editor}

\subsection{Version Control}

It was decided that the project source code should be managed using VCS
software, rather than do it manually. This had numerous advantages, including
keeping a record of the project's history, data backup and synchronisation
across computers. A number of tools were considered. One was
\emph{Subversion}\footnote{http://subversion.apache.org/}, a centralised version
control system. This was a natural choice as it is available on the lab
computers, and there is an svn server running on \emph{cosmos.ucc.ie}. However,
the flood of November 2009 and its effect on the availability of \emph{cosmos}
meant that an alternative needed to be found.

This presented an opportunity to try a modern distributed version control
system. The first one to be used was
\emph{Bazaar}\footnote{http://bazaar-vcs.org/}, which is closely associated with
Canonical Ltd.\ and the Ubuntu operating system. It has some commands in common
with Subversion, making it easy to transfer.

\emph{Git}\footnote{http://git-scm.com/} is another open source distributed
version control system, developed by Linus Torvalds in 2005 to track the source
code of the Linux kernel, replacing the proprietary \emph{Bitkeeper} system.

The Git repository for this project can be found at:
\begin{center}http://github.com/Dockheas23/scheme2lua\end{center}

\subsection{Typesetter}

This report was written and typeset using \LaTeX.

\subsection{Graphics, Diagrams \& Presentation}

Openoffice Draw, Dia and Openoffice Presentation


\section{Supervisor Meetings}

The supervisor meetings were held in Dr.\ Manning's office on a weekly basis,
usually on Thursday afternoons. The meetings generally had the following format:

\begin{enumerate}
\item An investigation of some specific problems encountered
\item A general discussion of the current state of the project, of the
features of Scheme and Lua, and the similarities and differences between
them
\item A look at the project direction and goals in high-level terms
\end{enumerate}

The frequency of the meetings helped the momentum of the project, and it meant
that potential pitfalls could be confronted and dealt with on a regular basis.
Revisiting and revising objectives every week made it flexibile and dynamic,
which facilitated exploration of specific areas when necessary.

