Project development took place from late October 2009 to late March 2010. As
the period progressed, there was a cumulative increasing focus on the project
work, as the workload from other assignments receded. Work was characterised by
regular meetings with the supervisor, with a number of periodic code
submissions.

Given that this was a reasonably small, single-person project, and somewhat
experimental in nature, formal project management methodologies were not
rigorously applied. Instead it was managed through constant attention, with the
help of some software tools.


\section{Methodology}

The project followed a mainly evolutionary lifecycle. It was also clear from the
beginning that it would not be possible to have a thoroughly complete translator
in the available time.  As a result, the project followed the course of a
prioritised list of objectives, which were followed in a serial fashion for the
most part.  Certain minor features were worked on in parallel, and the source
code management system facilitated that, but this was largely the exception for
two reasons: firstly each feature needed to be completed in full to avoid
breaking the entire program; and secondly, there was significant linear
dependency between features.


\section{Supervisor Meetings}

The supervisor meetings were held in Dr.\ Manning's office on a weekly basis,
usually on Thursday afternoons. The meetings generally had the following format:

\begin{enumerate}
\item An investigation of some specific problems encountered;
\item A general discussion of the current state of the project, of the
features of Scheme and Lua, and the similarities and differences between
them;
\item A look at the project direction and goals in high-level terms.
\end{enumerate}

The frequency of the meetings helped the momentum of the project, and it meant
that potential pitfalls could be confronted and dealt with on a regular basis.
Revisiting and revising objectives every week made it flexibile and dynamic,
which facilitated exploration of specific areas when necessary.


\section{Tools}

\subsection{Text Editor}

The \emph{vim}\footnote{\url{http://www.vim.org/}} editor was used to create and
edit all of the source code. Vim stands for Vi-Improved, a feature-rich version
of the old UNIX editor. It is widely available, versatile and highly
configurable. Among other things, it contains automatic indenting and syntax
highlighting making it more than adequate for a project such as this. In any
case, the volume of code didn't warrant a large Integrated Development
Environment.

\subsection{Version Control}

It was decided that the project source code should be managed using
VCS\footnote{Version Control System} software, rather than do it manually. This
had numerous advantages, including keeping a record of the project's history,
data backup and synchronisation across computers. A number of tools were
considered. One was
\emph{Subversion}\footnote{\url{http://subversion.apache.org/}}, a centralised
version control system. This was a natural choice as it is available on the lab
computers, and there is an svn repository running on \emph{cosmos.ucc.ie}.
However, the flood of November 2009 and its effect on the availability of
\emph{cosmos} meant that an alternative needed to be found.

This presented a great opportunity to try a modern distributed version control
system. The distributed model made more sense, particularly in this context, for
the following reasons:

\begin{description}
\item[Availability] \hfill \\
As mentioned above, the primary motivator for the move was the availability of
\emph{cosmos}, as it regularly had to be taken offline to service the generators
in the wake of the flood. 
\item[Data Backup] \hfill \\
Without a central server, there is no ``single point of failure''. Backing
up the entire history became simply a matter of working on the project from a
number of different computers. It was also possible to regularly synchronise
with an online code repository, for further redundancy.
\item[Workflows] \hfill \\
The range of possible workflows for a distributed VCS is greater than for a
centralised one. Being a single-person project, the centralised model is
unnecessary and prohibitively restrictive.
\item[Speed] \hfill \\
A distributed VCS doesn't suffer the network latency overheads of a centralised
system for the majority of its operations. Though this is not hugely significant
for such a small project, over time the delay adds up, so it was still a
consideration.
\item[Relevance] \hfill \\
There is an increasing trend in free and open-source projects towards
distributed version control, so this presented a chance to acquire skill in an
important contemporary tool.
\end{description}

The first distributed VCS to be used was
\emph{Bazaar}\footnote{\url{http://bazaar-vcs.org/}}, which is closely
associated with Canonical Ltd.\ and the Ubuntu operating system. It has some
commands in common with Subversion, making it easy to make the move. It was used
to track the project for a period from November to January, but it proved to be
a bit slow, and there was some difficulty getting it to work through the
University firewall.

At the beginning of February, the project was moved to
\emph{Git}\footnote{\url{http://git-scm.com/}}. Git is another open source
distributed version control system, developed by Linus Torvalds in 2005 to track
the source code of the Linux kernel, replacing the proprietary \emph{Bitkeeper}
system. It is more difficult to learn, but it is very neat in concept, and is
considerably faster than Bazaar for normal use.

The online Git repository for this project, which includes the project history
since 1st~February~2010, can be found at:
\begin{framed}
\centering
\url{http://github.com/Dockheas23/scheme2lua}
\end{framed}

\subsection{Typesetter}

This report was written and typeset using
\LaTeX\footnote{\url{http://www.latex-project.org/}}, with support for
generating pdf files. It was originally created by Leslie Lamport as a
convenient and usable set of macros for Donald Knuth's typesetting system, \TeX.
It has become a de-facto standard for scientific papers, and produces
well-structured and aesthetically pleasing documents.

\subsection{Graphics, Diagrams \& Presentation}

\emph{OpenOffice}\footnote{\url{http://www.openoffice.org/}} is a complete open
source office suite created by Sun Microsystems.  It can work with several
formats, including the standard OpenDocument format as well as Microsoft's
proprietary file formats. Its drawing and presentation programs were used to
create the diagrams and slides for the project open day.

Another open source program called
\emph{Dia}\footnote{\url{http://projects.gnome.org/dia/}} was used to create the
UML diagrams in later sections of this report. It is part of the
\emph{GNOME}\footnote{\url{http://www.gnome.org/}} project and can work with a
variety of different types of flow-charts and block diagrams, using a simple
drag-and-drop interface. It also contains an automatic code generation function,
and can export to a number of file formats, including png, pdf, eps and svg.

