Scheme and Lua are two languages with different origins. Scheme emerged in the
1970s as a dialect of Lisp, which in turn had been around since the late 1950s.
Lua was created in 1993 and was designed to be an extension language, for
configuring applications written in other languages. They are both small and
simple by design, and this makes them perfect candidates for research into their
possible associations.

The idea for the project came from Dr.\ Manning, and it was, to paraphrase his
introduction: {\em an exploration of the relationship between the languages
Scheme and Lua, using the creation of a source-to-source translator as the
means}. It was to be written in Lua, and would have the side-effect of being an
interesting new implementation of Scheme. Since both Scheme and Lua were unknown
to me, a further side-effect would be the knowledge of two new languages, with
all the nuances in programming style that accompany them.


\section{Why A Scheme To Lua Translator?}

Aside from the immediate answer, \emph{``Why not?''}, there are compelling
reasons for doing a project like this. Having to compare and contrast two
languages in the way that is necessary, has the potential to be far more
revealing than knowing either or both of them in isolation. It forces them to be
viewed outside the context of their traditional domains, and as such, fully
exposes their strengths and weaknesses.

Given the languages in question, it also compares two quite different syntaxes,
showing the paradigms that each tends towards, the boundaries of possible
expression, and conceivably the reasons why. Due to the compactness of their
cores, this can be done in an environment that is not overwhelmed by voluminous
manuals and library APIs.

The true objective of this project is to get to the very core of understanding
computer languages. As an aside, the topic transcends Computer Science to a
certain extent, entering the realm of all language and communication in general,
and any personal revelations from this would also be beneficial.

\section{Related Work}

There is no obvious evidence of another Scheme To Lua translator in existence.
However:
\begin{itemize}
\item Other source-to-source translators exist between different languages.
\item The translator has a certain degree of similarity to any compiler or
interpreter;
\item There was also a project in 2006, by a programmer named David
Bergman, to make an embedded Lisp interpreter, written in Lua~\cite{lualisp}.
\\[5mm]
\end{itemize}

\hrule \hfill \\[5mm]

\noindent The best way to begin the investigation is with a closer look at the
two languages in question.

\section{About Scheme}

\begin{quotation}
``Scheme is a statically scoped and properly tail-recursive
dialect of the Lisp programming language invented by Guy Lewis Steele Jr.\ and
Gerald Jay Sussman. It was designed to have an exceptionally clear and simple
semantics and few different ways to form expressions. A wide variety of
programming paradigms, including functional, imperative, and message passing
styles, find convenient expression in Scheme.''~\cite{r6rs}
\end{quotation}

\subsection{Scheme Usage}

Because of its simple syntax, Scheme is often used as the basis for introductory
Computer Science courses. It also acts as a scripting language for some free
software applications, such as the GNU Image Manipulation Program
(GIMP)\footnote{\url{http://www.gimp.org/}}, and the free accounting
application, GnuCash\footnote{\url{http://www.gnucash.org/}}

\subsection{Scheme Syntax}

Being a dialect of Lisp, Scheme has a Lisp-like syntax, which is very simple and
contains many parentheses. Here is an example taken from \emph{Scheme and the
Art of Programming}~\cite[p.40]{schemebook}
\begin{framed}
\begin{verbatim}
(define factorial
  (lambda (item)
    (cond
      ((pair? item) 'pair)
      ((null? item) 'empty-list)
      ((number? item) 'number)
      ((symbol? item) 'symbol)
      (else 'some-other-type))))
\end{verbatim}
\end{framed}


\section{About Lua}

\begin{quotation}
``Lua is an extension programming language designed to support general
procedural programming with data description facilities. It also offers good
support for object-oriented programming, functional programming, and data-driven
programming. Lua is intended to be used as a powerful, light-weight scripting
language for any program that needs one. Lua is implemented as a library,
written in clean C (that is, in the common subset of ANSI C and
C++).''~\cite[Sec~1]{luamanual}
\end{quotation}
\subsection{Lua Usage}

Lua is frequently embedded in other languages, but it can be used for
stand\-alone applications as well. Though originally developed as a scripting
language for engineers, it has found uses in many other industrial settings from
robotics and bioinformatics to image processing; and it has become extremely
popular for game programming~\cite{evolua}.

\subsection{Lua Syntax}

The syntax of Lua is quite different to that of Scheme, and has more in common
with languages like C, Java and others from the same lineage. The following example is from \emph{Programming in Lua}~\cite[p.40]{luabook}
\begin{framed}
\begin{verbatim}
function add (...)
  local s = 0
  for i, v in ipairs{...} do
    s = s + v
  end
  return s
end
\end{verbatim}
\end{framed}

\subsection{LuaJIT}

LuaJIT\cite{luajit} is a just-in-time compiler for Lua that, according to its
website\footnote{\url{http://luajit.org/performance.html}}, displays radical
improvements in performance over the regular Lua interpreter. This is another
interesting possibility when testing the performance of the translated output.


\section{Scheme \& Lua}

From the descriptions above, it would seem on the surface that Scheme and Lua
have very little in common. However, Scheme had, and continues to have, a
prominent influence in the development of Lua. The following extract from a
paper by Lua's creators, entitled \emph{``The Evolution of Lua''}, demonstrates this connection.

\begin{quotation}
''Semantically, Lua has many similarities with Scheme, even though these
similarities are not immediately clear because the two languages are
syntactically very different. The influence of Scheme on Lua has gradually
increased during Lua's evolution: initially, Scheme was just a language in the
background, but later it became increasingly important as a source of
inspiration, especially with the introduction of anonymous functions and full
lexical scoping.

Like Scheme, Lua is dynamically typed: variables do not have types; only values
have types. As in Scheme, a variable in Lua never contains a structured value,
only a reference to one. As in Scheme, a function name has no special status in
Lua: it is just a regular variable that happens to refer to a function
value.''\ldots ``Like Scheme, Lua has first-class functions with lexical
scoping.''~\cite[Sec~2]{evolua}
\end{quotation}

\label{sec:listandtable}
Another parallel is the fact that both languages are based around a single data
structure. In Scheme, this is the \emph{list}, and it is integral to the core
syntax of all Scheme programs. Lists are composed of a series of pairs, as in a
generic linked list data structure:
\begin{quotation}
``The pair, or cons cell, is the most fundamental of Scheme's structured object
types. The most common use for pairs is to build lists, which are ordered
sequences of pairs linked one to the next by the cdr field. The elements of the
list occupy the car fields of the pairs. The cdr of the last pair in a proper
list is the empty list, (); the cdr of the last pair in an improper list can be
anything other than ().''~\cite[Sec~6.4]{tspl}
\end{quotation}

In Lua, it is the \emph{table}, a flexible map-like data structure. The power of
tables allows many things to be accomplished, including a kind of
object-oriented programming: 
\begin{quotation}
``Tables are the main (in fact, the only) data structuring mechanism in Lua, and
a powerful one. We use tables to represent ordinary arrays, symbol tables, sets,
records, queues, and other data structures, in a simple, uniform, and efficient
way''~\cite[p.14]{luabook}.
\end{quotation}

This prominence of a single data structure in both languages could be a useful
factor in building the translator.
