As stated in the beginning, the underlying aim of this project was to explore
two languages, using a translator as the means. The revelations from this
exploration were many, and in a number of areas, from the intricacies of
programming syntax to the challenges of managing a project of several months in
duration.

\section{Learning Outcomes}

From a design perspective, the major lesson was the compromise that needed to
be reached between the translator itself and its output, and the cost of a
change in direction. In hindsight, making a large design change during the
project was probably a mistake, both in terms of the time it consumed in
development and testing, and the likely effect that the result had on
performance. It simplified the translator at the expense of the size,
readability and complexity of the translated program. More priority should have
been given to the translator output, and a less radical solution should have
been found.

In terms of its real purpose, the project provided a unique and captivating view
of Scheme and Lua, and the relationships between them. It exposed the
interesting melange of programming paradigms and styles they are capable of. It
revealed what effect syntax has on what can be programmed, and in what way. And
it also gave a deeper understanding of programming languages in general, that
can be applied to all future interactions with all languages.

\section{Further Work}

As already described in Section~\ref{sec:outstandingfeatures}, these are the
main areas for further enhancement of the translator. Here they are ordered
using a subjective scale, beginning with the ``most interesting'':
\begin{enumerate}
\item Continuations
\item Syntax Extension
\item The Scheme Number Syntax
\item User-Defined Identifiers
\item Equality
\item Full coverage of Scheme procedures
\end{enumerate}


\section{A Final Word}

Creating the translator was intensely enjoyable and challenging. It was an
unrivalled opportunity for me to have fun with computer languages, learning two
great new ones along the way; but knowing that this project was of particular
interest to the supervisor, I also really wanted to do it justice. The project
has developed and shaped my perspective on Computer Science immeasurably, and in
the most gratifying way. If I had my time again I'd do a number of things
differently, and that is the measure of a great learning experience. I'd like to
thank Dr.\ Manning for providing inspiration, guidance, great advice, and an
utterly fascinating project.
